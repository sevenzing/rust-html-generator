\chapter{Introduction}
\label{chap:intro}

\section{Background and Motivation}

Software development is a complex task that involves writing and maintaining large codebases. As projects grow in size and complexity, it becomes increasingly difficult for developers to understand the code and identify potential issues. One of the primary challenges that developers face is the need to comprehend programs quickly and accurately. This is particularly challenging when working with code that was written a long time ago by different people and lacks proper documentation.

To address this challenge, developers have turned to various tools and programs that can help them comprehend and write code more conveniently. These tools are designed to make the development process more efficient by providing developers with the information they need to navigate the project effectively.

One of the most popular and useful approaches to this problem is text program visualization. This approach is used in many IDEs, code viewers, and documentation generators and is characterized by several key features. One of the most important features of text program visualization is syntax highlighting. This feature involves highlighting different parts of the code in different colors to make it easier for developers to identify keywords, functions, and other important elements of the code. This makes it easier to scan through the code and identify potential issues quickly.

Another important feature of text program visualization is the use of cross-entity links. These links allow developers to move easily between different parts of the project, such as functions, definitions, and other elements. This makes it easier for developers to navigate the codebase and understand the relationships between different parts of the code.

In addition to these features, text program visualization tools also provide information about the declaration and usage of variables and other elements of the code. This information can be critical for understanding how the code works and identifying potential issues that may arise.

Overall, text program visualization has become an essential tool for software developers. It provides a powerful and efficient way to comprehend and write code more conveniently. By leveraging the features of text program visualization, developers can work more efficiently and effectively, reducing the time and effort required to develop and maintain complex projects.

\section{Task description}

The aim of this thesis is to develop an autonomous tool that will analyze modern and popular Rust programming language and convert it into a portable \textit{HTML} format, which can be easily opened with any popular web browser. This tool is intended to provide an extensive code visualization functionality to the users, which will help them to quickly understand and comprehend the Rust code.

The task involves developing a software tool that will analyze the Rust code, extract the relevant information, and present it in a visually appealing and easy-to-understand format. The tool should be able to handle large and complex codebases, which may have been written by different people, and may not have any documentation.

To achieve this goal, the tool will leverage the popular and useful features of existing solutions in the field of Rust code text visualization, such as syntax highlighting, cross-entity links, and displaying information about declaration and usage. However, unlike existing solutions, our tool will focus on providing cross-platform portability by generating a report in the form of a single HTML file.

The tool will be mainly used by developers who need to quickly understand the code, such as a supervisor of an intern who needs to review the code, a person who just joined a large company and needs to understand the project as quickly as possible, or a customer who wants to see the result of the project by looking at the internal code implementation.

The development of this tool involves several stages, including research on Rust programming language, analyzing the existing solutions for Rust code text visualization, designing the architecture of the tool, implementing the tool, and testing it for various use cases. The final outcome of this thesis is fully functional tool that can be used by developers to understand and comprehend the Rust code in a more convenient and efficient way.

\section{Applicability}

The application developed in this thesis will have numerous applications in the field of Rust development. First and foremost, it will serve as a valuable tool for developers who need to understand the structure and workings of a Rust project. By providing a detailed and navigable HTML report, the application will make it easier for developers to explore the codebase and identify potential issues.

The application will also be useful for project managers and other stakeholders who need to review Rust projects. By providing a clear and concise overview of the codebase, the report will allow these individuals to assess the project's progress and identify any areas that require further attention.

Finally, the application will be useful for researchers and educators who need to study the Rust programming language. By providing a detailed and navigable HTML report of a Rust project, the application will make it easier for these individuals to explore the language and understand its various elements. Additionally, the application can be used as a teaching tool to introduce students to Rust programming and provide them with hands-on experience exploring a real-world project.

